%\documentclass{article}
%\usepackage{lscape}
%\begin{document}

\begin{landscape}
\begin{table}
\centering
\caption{Skin and Fleece Characteristics which may Reflect Stages in Merino Evolution}
\label{tb:2}
\vspace{0.1in}
\begin{tabular}{p{0.7in}|p{0.6in}|p{0.6in}|l|l|p{0.7in}|l|p{1.0in}|p{1.2in}}  \hline

 Line of evolution &  Approx. time scale & Congenetic sheep &  Dp (microns) &  Ds (microns) & Medullation  &  S/P Ratio &  Follicle Group Arrangement &   Fleece structure \\ \hline \hline

Wild Sheep &  9000 B.C. & Mouflon, Barbary & 150  &  15  &  latticed &  3-5 & S between P, P  in straight lines & Two coated, long medullated guard hairs and fine underwool \\ \hline

Primitive Domestic Sheep & 3000-1000 B.C. &  Soay, Asiatic & 42 & 17 & non-latticed, continuous &  4-5  & S in two groups, point of wedge between P & Ill defined staples with curly tips and fine fibres \\ \hline

Ancient Fine to Medium Wool &  500 B.C.- &  Dead sea scroll material, ancient textiles & 38 & 21 & interrupted & 5-7 & S wedges merged, P closer and in staples?    &  Heterotype hairs, fine/medium fibres \\ \hline

True Fine Wool & 1500-1850 A.D. &  Spanish Merino & 19-24 & 17-21 & nil? & 20 & So further from P, Sd between So and P& Well defined wool staples, blocky tips, fibres uniformly fine/medium diameter and uniform length \\ \hline

Australian Merino &  C.1830-1988 A.D. & & & & & & & \\
  Fine    &  & &  16-22 &  16-21 &   -  &  16-24 &  - &  - \\
  Medium  &  & &  21-29 &  16-24 &   -  &  19-27 &  - &  - \\
  Strong  &  & &  29-32 &  22-25 &   -  &  15-18 &  - &  - \\ 
  Other ? & & & & & & & & \\ \hline

\end{tabular}
\end{table}
\end{landscape}

%\end{document}
