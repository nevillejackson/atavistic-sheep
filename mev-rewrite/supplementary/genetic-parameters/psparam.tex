%
% Draft  document psparam.tex
% Summarizes best available estimates of heritabilities and genetic correlations
% for Dp, Ds, Np, Ns, Ns/Np, and some other wool traits.
%
 
\documentclass[titlepage]{article}  % Latex2e
\usepackage{graphicx,lscape,subfigure}
\usepackage{bm,longtable}
\usepackage{textcomp}
 

\title{Genetics of primary and secondary fibre diameters and densities in Merino sheep}
\author{Neville Jackson}
\date{18 Sep 2017} 

 
\begin{document} 


 
\maketitle      
\tableofcontents

$\newcommand{\E}{\mathrm{E}}$
$\newcommand{\Var}{\mathrm{Var}}$
$\newcommand{\Cov}{\mathrm{Cov}}$ 
$\newcommand{\SD}{\mathrm{SD}}$ 

\clearpage
\section{Introduction} 
In the document Jackson etal(1990)~\cite{jack:90} some genetic parameter estimates were presented ( see Table 8 of Jackson etal(1990)~\cite{jack:90}) for diameters of primary and secondary fibres and a number of  other important wool characteristics. These were preliminary estimates obtained by the now outdated analysis of variance method for partitioning variances between and within half-sib families. 

A modern method of obtaining these parameters would be to estimate additive genetic variance components using maximum likelihood methods and incorating the full additive genetic relationship matrix  among all individuals, computed from a pedigree going back to the base generation of the flock. 

This analysis has been done on the exact same dataset for a comprehensive set of 48 traits (Jackson (2015)~\cite{jack:15}). What we present here is parameter estimates for a small subset of these traits which are relevant to the issues of Merino evolution and atavism.

\section{Materials and methods}
The sheep and measurements were as described in Jackson (2015)~\cite{jack:15}.
\subsection{Traits}
The subset of traits covered here are summarized in Table~\ref{tab:traits}
%\documentclass{article}
%\usepackage{lscape,longtable}
%\begin{document}
\begin{center}
\begin{landscape}
\begin{longtable}{p{1.5in}|p{0.8in}|p{1.5in}|p{1.0in}|p{2.5in}}
\caption{Definition of traits measured}  \\
\hline
\label{tab:traits}
    Trait name & Abbreviation  & Units & Age measured  &  Description \\ 
\hline
\endfirsthead
\multicolumn{5}{c}%
{\tablename\ \thetable\ -- \textit{Continued from previous page}} \\
\hline
    Trait name & Abbreviation  & Units & Age measured  &  Description \\ 
\hline
\endhead
\hline
\multicolumn{5}{r}{\textit{Continued on next page}} \\
\endfoot
\hline
\endlastfoot
%\env{longtable}[p{1.5in}|p{0.8in}|p{1.5in}|p{1.0in}|p{2.5in}]
 Clean wool weight & Cww & Kg & 14 months & Weight of clean fibre at 16\% regain \\
 Follicle number per unit area & Fnua & no per $mm_{2}$ & 14 months & No of primary and secondary follicles per $mm_{2}$ from skin biopsy \\
 Follicle $S/P$ ratio & Fr & no units & 14 months & Ratio of no of primary to no of secondary follicles from skin biopsy \\
  Birthcoat score back & Bctb & score 1-6 (1=no halo hairs on mid backline, 6=dense halo hairs) & day of birth & Score for density of halo hairs on mid backline on day of birth \\
  No of lambs born & NLB & no & day of birth & Number of lambs in litter at birth \\
  No of lambs weaned & NLW & no & approx 4 months & Number of lambs in litter at weaning \\
  Mean diameter of primaries & Dp & microns & 14 months & Mean diameter of primary fibres from biopsy and horizontal section \\
  Mean diameter of secondaries & Ds & microns & 14 months & Mean diameter of secondary fibres from biopsy and horizontal section \\
  Mean diameter of primaries and secondaries & Dps & microns & 14 months & Mean diameter of primary and secondary fibres from biopsy and horizontal section \\
  Primary to secondary diameter ratio & DpovDs & no units & 14 months & Ratio of mean diameter of primary fibres to mean diameter of secondary fibres \\
  SD of primaries & SDDp & microns & 14 months & Standard deviation of primary fibre diameter \\
  SD of secondaries & SDDs & microns & 14 months & Standard deviation of secondary fibre diameter \\
  SD of all fibres & SDD & microns & 14 months & Standard deviation of primary and secondary fibre diameter \\

\end{longtable}
\end{landscape}
\end{center}
%\end{document}


\subsection{Statistical Methods}
The statistical techniques used for estimation of parameters for these data are covered in Jackson (2015)~\cite{jack:15}.  The software used is called {\em dmm}, which runs as a package under the R statistical language ~\cite{}. {\em dmm} is documented in a user's guide (Jackson(2015b)~\cite{jack:15b}).

\section{Results}
All parameter estimates are presented with a standard error estimate and the upper and lower bounds of a 95 percent confidence interval.
\subsection{heritabilities}
Estimates of heritability for the 13 traits are given in Table~\ref{tab:herit}
% latex table generated in R 3.2.4 by xtable 1.8-2 package
% Wed Sep 20 19:59:27 2017
%\documentclass{article}
%\usepackage{lscape,longtable}
%\begin{document}
\begin{center}
\begin{longtable}{|p{0.9in}|p{0.7in}|p{0.7in}|p{0.7in}|p{0.7in}|}
\caption{Heritability estimates with standard errors and 95 percent confidence limits for 13 skin and wool traits} \\
\hline
\label{tab:herit}

Trait  & Estimate & StdErr & CI95lo & CI95hi \\ 
  \hline
Cww  & 0.34 & 0.01 & 0.32 & 0.37 \\ 
  Fnua  & 0.29 & 0.01 & 0.27 & 0.32 \\ 
  Fr &  0.31 & 0.01 & 0.28 & 0.33 \\ 
  Bctb &  0.76 & 0.02 & 0.72 & 0.79 \\ 
  NLB &  0.20 & 0.01 & 0.18 & 0.22 \\ 
  NLW &  0.21 & 0.01 & 0.19 & 0.23 \\ 
  Dp &  0.52 & 0.03 & 0.45 & 0.58 \\ 
  Ds &  0.41 & 0.03 & 0.35 & 0.47 \\ 
  Dps &  0.38 & 0.03 & 0.33 & 0.44 \\
  DpovDs &  0.81 & 0.04 & 0.74 & 0.88 \\ 
  SDDp &  0.49 & 0.03 & 0.43 & 0.55 \\ 
  SDDs &  0.41 & 0.03 & 0.35 & 0.47 \\ 
  SDD &  0.43 & 0.03 & 0.37 & 0.48 \\ 
   \hline

\end{longtable}
\end{center}
%\end{document}


The heritability estimates for density and ratio are somewhat different from those of Jackson, Nay, and Turner (1975)~\cite{jack:75}. Estimates for Fnua and Fr are lower than previous estimates. However  for Cww and Dps estimates agree with other published values ( see Turner and Young (1969)~\cite{turn:69}). The high estimates for Bctb and DpovDs are surprising, but given that estimates for other traits are reasonable , we have to accept them. The estimates for Dp and Ds and their standard deviations are new - there are no other published values, but their magnitude looks reasonable given that most wool traits have about a 40 perent heritability. We would expect lower values for NLB and NLW so the 20 percent values are quite acceptable.

\subsection{genetic correlations}
The additive genetic correlation estimates for all pairs of the 13 traits are presented in Table~\ref{tab:rgen}.
% latex table generated in R 3.2.4 by xtable 1.8-2 package
% Wed Sep 20 20:44:52 2017
%\documentclass{article}
%\usepackage{lscape,longtable}
%\begin{document}
\begin{center}
\begin{longtable}{|p{0.9in}|p{0.7in}|p{0.7in}|p{0.7in}|p{0.7in}|}
\caption{Genetic correlation estimates with standard errors and 95 percent confidence limits for 13 skin and wool traits}  \\
\hline
\label{tab:rgen}

  Traits & Estimate & StdErr & CI95lo & CI95hi \\ 
  \hline
\endfirsthead
\multicolumn{5}{c}%
{\tablename\ \thetable\ -- \textit{Continued from previous page}} \\
\hline
  Traits & Estimate & StdErr & CI95lo & CI95hi \\
\hline
\endhead
\hline
\multicolumn{5}{r}{\textit{Continued on next page}} \\
\endfoot
\hline
\endlastfoot

Cww:Cww & 1.00 & 0.00 & 1.00 & 1.00 \\ 
  Cww:Fnua & 0.01 & 0.04 & -0.07 & 0.08 \\ 
  Cww:Fr & 0.06 & 0.04 & -0.01 & 0.13 \\ 
  Cww:Bctb & -0.12 & 0.03 & -0.18 & -0.07 \\ 
  Cww:NLB & 0.15 & 0.04 & 0.07 & 0.24 \\ 
  Cww:NLW & 0.10 & 0.04 & 0.02 & 0.18 \\ 
  Cww:Dp & -0.01 & 0.06 & -0.12 & 0.10 \\ 
  Cww:Ds & 0.63 & 0.06 & 0.50 & 0.75 \\ 
  Cww:Dps & 0.63 & 0.06 & 0.50 & 0.75 \\ 
  Cww:DpovDs & -0.31 & 0.04 & -0.40 & -0.22 \\ 
  Cww:SDDp & -0.07 & 0.06 & -0.18 & 0.04 \\ 
  Cww:SDDs & 0.30 & 0.06 & 0.18 & 0.43 \\ 
  Cww:SDD & 0.27 & 0.06 & 0.14 & 0.39 \\ 
  Fnua:Cww & 0.01 & 0.04 & -0.07 & 0.08 \\ 
  Fnua:Fnua & 1.00 & 0.00 & 1.00 & 1.00 \\ 
  Fnua:Fr & 0.49 & 0.03 & 0.43 & 0.56 \\ 
  Fnua:Bctb & 0.07 & 0.03 & 0.02 & 0.13 \\ 
  Fnua:NLB & 0.15 & 0.05 & 0.05 & 0.25 \\ 
  Fnua:NLW & 0.22 & 0.05 & 0.11 & 0.32 \\ 
  Fnua:Dp & 0.00 & 1.34 & -2.63 & 2.63 \\ 
  Fnua:Ds & -0.73 & 0.06 & -0.85 & -0.61 \\ 
  Fnua:Dps & -0.74 & 0.06 & -0.86 & -0.62 \\ 
  Fnua:DpovDs & 0.36 & 0.06 & 0.26 & 0.47 \\ 
  Fnua:SDDp & -0.06 & 0.07 & -0.19 & 0.08 \\ 
  Fnua:SDDs & -0.37 & 0.07 & -0.52 & -0.23 \\ 
  Fnua:SDD & -0.36 & 0.07 & -0.51 & -0.22 \\ 
  Fr:Cww & 0.06 & 0.04 & -0.01 & 0.13 \\ 
  Fr:Fnua & 0.49 & 0.03 & 0.43 & 0.56 \\ 
  Fr:Fr & 1.00 & 0.00 & 1.00 & 1.00 \\ 
  Fr:Bctb & 0.02 & 0.03 & -0.04 & 0.07 \\ 
  Fr:NLB & 0.08 & 0.05 & -0.02 & 0.18 \\ 
  Fr:NLW & 0.12 & 0.05 & 0.02 & 0.22 \\ 
  Fr:Dp & -0.13 & 0.07 & -0.27 & 0.01 \\ 
  Fr:Ds & -0.12 & 0.08 & -0.28 & 0.03 \\ 
  Fr:Dps & -0.15 & 0.08 & -0.31 & 0.01 \\ 
  Fr:DpovDs & -0.03 & 0.06 & -0.15 & 0.08 \\ 
  Fr:SDDp & -0.12 & 0.07 & -0.27 & 0.02 \\ 
  Fr:SDDs & -0.27 & 0.08 & -0.43 & -0.11 \\ 
  Fr:SDD & -0.28 & 0.08 & -0.44 & -0.12 \\ 
  Bctb:Cww & -0.12 & 0.03 & -0.18 & -0.07 \\ 
  Bctb:Fnua & 0.07 & 0.03 & 0.02 & 0.13 \\ 
  Bctb:Fr & 0.02 & 0.03 & -0.04 & 0.07 \\ 
  Bctb:Bctb & 1.00 & 0.00 & 1.00 & 1.00 \\ 
  Bctb:NLB & 0.06 & 0.03 & 0.00 & 0.12 \\ 
  Bctb:NLW & 0.17 & 0.03 & 0.12 & 0.23 \\ 
  Bctb:Dp & 0.70 & 0.03 & 0.65 & 0.76 \\ 
  Bctb:Ds & -0.14 & 0.04 & -0.22 & -0.07 \\ 
  Bctb:Dps & -0.08 & 0.04 & -0.16 & -0.00 \\ 
  Bctb:DpovDs & 0.65 & 0.02 & 0.61 & 0.69 \\ 
  Bctb:SDDp & 0.85 & 0.03 & 0.79 & 0.91 \\ 
  Bctb:SDDs & 0.31 & 0.04 & 0.24 & 0.38 \\ 
  Bctb:SDD & 0.43 & 0.04 & 0.36 & 0.50 \\ 
  NLB:Cww & 0.15 & 0.04 & 0.07 & 0.24 \\ 
  NLB:Fnua & 0.15 & 0.05 & 0.05 & 0.25 \\ 
  NLB:Fr & 0.08 & 0.05 & -0.02 & 0.18 \\ 
  NLB:Bctb & 0.06 & 0.03 & 0.00 & 0.12 \\ 
  NLB:NLB & 1.00 & 0.00 & 1.00 & 1.00 \\ 
  NLB:NLW & 0.98 & 0.03 & 0.93 & 1.03 \\ 
  NLB:Dp & 0.41 & 0.08 & 0.26 & 0.55 \\ 
  NLB:Ds & -0.16 & 0.09 & -0.33 & 0.02 \\ 
  NLB:Dps & -0.12 & 0.09 & -0.30 & 0.06 \\ 
  NLB:DpovDs & 0.42 & 0.06 & 0.30 & 0.54 \\ 
  NLB:SDDp & 0.38 & 0.08 & 0.22 & 0.54 \\ 
  NLB:SDDs & 0.06 & 0.09 & -0.10 & 0.23 \\ 
  NLB:SDD & 0.14 & 0.08 & -0.03 & 0.30 \\ 
  NLW:Cww & 0.10 & 0.04 & 0.02 & 0.18 \\ 
  NLW:Fnua & 0.22 & 0.05 & 0.11 & 0.32 \\ 
  NLW:Fr & 0.12 & 0.05 & 0.02 & 0.22 \\ 
  NLW:Bctb & 0.17 & 0.03 & 0.12 & 0.23 \\ 
  NLW:NLB & 0.98 & 0.03 & 0.93 & 1.03 \\ 
  NLW:NLW & 1.00 & 0.00 & 1.00 & 1.00 \\ 
  NLW:Dp & 0.42 & 0.07 & 0.28 & 0.57 \\ 
  NLW:Ds & -0.23 & 0.09 & -0.40 & -0.06 \\ 
  NLW:Dps & -0.19 & 0.09 & -0.36 & -0.02 \\ 
  NLW:DpovDs & 0.47 & 0.06 & 0.36 & 0.59 \\ 
  NLW:SDDp & 0.42 & 0.08 & 0.27 & 0.57 \\ 
  NLW:SDDs & 0.22 & 0.08 & 0.06 & 0.38 \\ 
  NLW:SDD & 0.28 & 0.08 & 0.12 & 0.44 \\ 
  Dp:Cww & -0.01 & 0.06 & -0.12 & 0.10 \\ 
  Dp:Fnua & 0.00 & 1.34 & -2.63 & 2.63 \\ 
  Dp:Fr & -0.13 & 0.07 & -0.27 & 0.01 \\ 
  Dp:Bctb & 0.70 & 0.03 & 0.65 & 0.76 \\ 
  Dp:NLB & 0.41 & 0.08 & 0.26 & 0.55 \\ 
  Dp:NLW & 0.42 & 0.07 & 0.28 & 0.57 \\ 
  Dp:Dp & 1.00 & 0.00 & 1.00 & 1.00 \\ 
  Dp:Ds & -0.09 & 0.06 & -0.22 & 0.03 \\ 
  Dp:Dps & -0.00 & 0.15 & -0.30 & 0.29 \\ 
  Dp:DpovDs & 0.87 & 0.02 & 0.82 & 0.91 \\ 
  Dp:SDDp & 0.85 & 0.03 & 0.78 & 0.92 \\ 
  Dp:SDDs & 0.34 & 0.05 & 0.24 & 0.45 \\ 
  Dp:SDD & 0.47 & 0.05 & 0.38 & 0.57 \\ 
  Ds:Cww & 0.63 & 0.06 & 0.50 & 0.75 \\ 
  Ds:Fnua & -0.73 & 0.06 & -0.85 & -0.61 \\ 
  Ds:Fr & -0.12 & 0.08 & -0.28 & 0.03 \\ 
  Ds:Bctb & -0.14 & 0.04 & -0.22 & -0.07 \\ 
  Ds:NLB & -0.16 & 0.09 & -0.33 & 0.02 \\ 
  Ds:NLW & -0.23 & 0.09 & -0.40 & -0.06 \\ 
  Ds:Dp & -0.09 & 0.06 & -0.22 & 0.03 \\ 
  Ds:Ds & 1.00 & 0.00 & 1.00 & 1.00 \\ 
  Ds:Dps & 1.00 & 0.00 & 0.99 & 1.00 \\ 
  Ds:DpovDs & -0.58 & 0.04 & -0.66 & -0.49 \\ 
  Ds:SDDp & -0.06 & 0.06 & -0.18 & 0.07 \\ 
  Ds:SDDs & 0.60 & 0.06 & 0.48 & 0.71 \\ 
  Ds:SDD & 0.54 & 0.06 & 0.42 & 0.65 \\ 
  Dps:Cww & 0.63 & 0.06 & 0.50 & 0.75 \\ 
  Dps:Fnua & -0.74 & 0.06 & -0.86 & -0.62 \\ 
  Dps:Fr & -0.15 & 0.08 & -0.31 & 0.01 \\ 
  Dps:Bctb & -0.08 & 0.04 & -0.16 & -0.00 \\ 
  Dps:NLB & -0.12 & 0.09 & -0.30 & 0.06 \\ 
  Dps:NLW & -0.19 & 0.09 & -0.36 & -0.02 \\ 
  Dps:Dp & -0.00 & 0.15 & -0.30 & 0.29 \\ 
  Dps:Ds & 1.00 & 0.00 & 0.99 & 1.00 \\ 
  Dps:Dps & 1.00 & 0.00 & 1.00 & 1.00 \\ 
  Dps:DpovDs & -0.50 & 0.05 & -0.59 & -0.41 \\ 
  Dps:SDDp & 0.02 & 0.06 & -0.10 & 0.14 \\ 
  Dps:SDDs & 0.63 & 0.06 & 0.51 & 0.75 \\ 
  Dps:SDD & 0.59 & 0.06 & 0.47 & 0.70 \\ 
  DpovDs:Cww & -0.31 & 0.04 & -0.40 & -0.22 \\ 
  DpovDs:Fnua & 0.36 & 0.06 & 0.26 & 0.47 \\ 
  DpovDs:Fr & -0.03 & 0.06 & -0.15 & 0.08 \\ 
  DpovDs:Bctb & 0.65 & 0.02 & 0.61 & 0.69 \\ 
  DpovDs:NLB & 0.42 & 0.06 & 0.30 & 0.54 \\ 
  DpovDs:NLW & 0.47 & 0.06 & 0.36 & 0.59 \\ 
  DpovDs:Dp & 0.87 & 0.02 & 0.82 & 0.91 \\ 
  DpovDs:Ds & -0.58 & 0.04 & -0.66 & -0.49 \\ 
  DpovDs:Dps & -0.50 & 0.05 & -0.59 & -0.41 \\ 
  DpovDs:DpovDs & 1.00 & 0.00 & 1.00 & 1.00 \\ 
  DpovDs:SDDp & 0.74 & 0.03 & 0.68 & 0.80 \\ 
  DpovDs:SDDs & -0.04 & 0.05 & -0.13 & 0.06 \\ 
  DpovDs:SDD & 0.10 & 0.05 & 0.01 & 0.19 \\ 
  SDDp:Cww & -0.07 & 0.06 & -0.18 & 0.04 \\ 
  SDDp:Fnua & -0.06 & 0.07 & -0.19 & 0.08 \\ 
  SDDp:Fr & -0.12 & 0.07 & -0.27 & 0.02 \\ 
  SDDp:Bctb & 0.85 & 0.03 & 0.79 & 0.91 \\ 
  SDDp:NLB & 0.38 & 0.08 & 0.22 & 0.54 \\ 
  SDDp:NLW & 0.42 & 0.08 & 0.27 & 0.57 \\ 
  SDDp:Dp & 0.85 & 0.03 & 0.78 & 0.92 \\ 
  SDDp:Ds & -0.06 & 0.06 & -0.18 & 0.07 \\ 
  SDDp:Dps & 0.02 & 0.06 & -0.10 & 0.14 \\ 
  SDDp:DpovDs & 0.74 & 0.03 & 0.68 & 0.80 \\ 
  SDDp:SDDp & 1.00 & 0.00 & 1.00 & 1.00 \\ 
  SDDp:SDDs & 0.37 & 0.06 & 0.26 & 0.48 \\ 
  SDDp:SDD & 0.52 & 0.05 & 0.42 & 0.61 \\ 
  SDDs:Cww & 0.30 & 0.06 & 0.18 & 0.43 \\ 
  SDDs:Fnua & -0.37 & 0.07 & -0.52 & -0.23 \\ 
  SDDs:Fr & -0.27 & 0.08 & -0.43 & -0.11 \\ 
  SDDs:Bctb & 0.31 & 0.04 & 0.24 & 0.38 \\ 
  SDDs:NLB & 0.06 & 0.09 & -0.10 & 0.23 \\ 
  SDDs:NLW & 0.22 & 0.08 & 0.06 & 0.38 \\ 
  SDDs:Dp & 0.34 & 0.05 & 0.24 & 0.45 \\ 
  SDDs:Ds & 0.60 & 0.06 & 0.48 & 0.71 \\ 
  SDDs:Dps & 0.63 & 0.06 & 0.51 & 0.75 \\ 
  SDDs:DpovDs & -0.04 & 0.05 & -0.13 & 0.06 \\ 
  SDDs:SDDp & 0.37 & 0.06 & 0.26 & 0.48 \\ 
  SDDs:SDDs & 1.00 & 0.00 & 1.00 & 1.00 \\ 
  SDDs:SDD & 0.99 & 0.01 & 0.97 & 1.00 \\ 
  SDD:Cww & 0.27 & 0.06 & 0.14 & 0.39 \\ 
  SDD:Fnua & -0.36 & 0.07 & -0.51 & -0.22 \\ 
  SDD:Fr & -0.28 & 0.08 & -0.44 & -0.12 \\ 
  SDD:Bctb & 0.43 & 0.04 & 0.36 & 0.50 \\ 
  SDD:NLB & 0.14 & 0.08 & -0.03 & 0.30 \\ 
  SDD:NLW & 0.28 & 0.08 & 0.12 & 0.44 \\ 
  SDD:Dp & 0.47 & 0.05 & 0.38 & 0.57 \\ 
  SDD:Ds & 0.54 & 0.06 & 0.42 & 0.65 \\ 
  SDD:Dps & 0.59 & 0.06 & 0.47 & 0.70 \\ 
  SDD:DpovDs & 0.10 & 0.05 & 0.01 & 0.19 \\ 
  SDD:SDDp & 0.52 & 0.05 & 0.42 & 0.61 \\ 
  SDD:SDDs & 0.99 & 0.01 & 0.97 & 1.00 \\ 
  SDD:SDD & 1.00 & 0.00 & 1.00 & 1.00 \\ 
   \hline

\end{longtable}
\end{center}
%\end{document}


The genetic correlations involving Dp are most relevent. Dp is not genetically correlated with either Ds or Cww or Fnua.  This indicates that Dp and Ds can be changed independently. Dp is strongly genetically correlated with Bcts, SDDp, and DpovDs. Dp also has a moderate genetic correlation with the fertility traits NLB and NLW.  This supports the argument that Dp  is in some way related to fitness.

The genetic correlations involving Ds are different to those involving Dp. Ds has a strong positive genetic correlation with Cww, and a strong negative genetic correlation with Fnua. Selection for Cww will increase Ds, but not Dp. Ds has weak negative genetic correlations with Bctb and the fertility traits (NLB and NLW). Ds is positively correlated with its standard deviation (SDDs).

The other correlations of interest are between Cww and SDDp, which is close to zero, and between Cww and SDDs which is medium positive. So Cww selection will change both the mean (Ds) and standard deviation(SDDs) of the secondary fibres, but the primary fibres will not be affected in either mean (Dp) or standard deviation (SDDp).  This goes against what has been said (Jackson et al (1990)~\cite{jack:90}) about Cww selection possibly being one of the factors which can cause primitive individuals to appear in a flock.  What the parameters say is that Cww selection without amendments will lead to coarse fibres and a broader distribution of diameter, but this will come from the secondary fibres getting coarser, not the primaries. Other factors must be involved when we observe individuals with coarse primary fibres. It would seem, from the parameters, that these other factors might be fitness related, because the present genetic correlations suggest that sheep with coarse primaries are more fertile and better survivors.

We can summarize the Cww selection issue as follows. There are two ways wool quality can be allowed to deteriorate while selecting a Fine Merino flock for Cww
\begin{itemize}
\item  failing to put sufficient counter-selection pressure on Ds and SDDs ( or D and SDD which is nearly the same thing). This leads to sheep becoming like Btritish Longwools - all fibres coarser. This is the components issue. The South Australian Merino is a good example.
\item failing to put sufficient counter-selection pressure on Dp to counter the fitness effect of Dp, and allowing density to decline (see below). This leads to sheep becoming like primitive 2-coated domestic sheep with Dp coarser than Ds. 
\end{itemize}

So the genetic correlations between Cww and Ds ( and between Cww and SDDs) are simply the positive correlations between Cww and and two of its components. Nothing else is involved, it is purely a physical part/whole situation.

But the genetic correlation between Cww and Dp is more complicated. First there is the part/whole components situation, which would tend to make the correlation positive. Second there is the fact that another component of Cww - density or Ns/Np ratio - has a strong negative correlation with Dp because the pre-papilla cell theory (Moore et al (1998)~\cite{moor:98}) shows that small primary follicles conserve pre-papilla cell numbers so that they can multiply more and form more secondary follicles. So this situation would tend to make the correlation between Cww and Dp negative, to the extent that variation in Cww involves variation in density. Third Dp is a fitness component because it is strongly correlated with birthcoat and fertility. If Cww is related to fitness ( one would expect low Cww to be fitter in a non-domestic environment) this would also lead to a negative correlation between Cww and Dp. The balance between these three effects is what we see as the genetic correlation between Cww and Dp. In our flock the balance was close to zero. In other flocks the balance could be different.

The net result of situations like the above where genetic correlations depend on a balancing of forces, is that predictions made using such genetic correlations are not likely to be very useful. The balance can change in a population under selection.

\subsection{Phenotypic variances and covariances}
To use genetic parameters for predicting selection response requires phenotypic (co)variances in addition to heritabilities and genetic correlations. We present these here for completeness in Table~\ref{tab:sigp}
% latex table generated in R 3.2.4 by xtable 1.8-2 package
% Thu Sep 21 20:05:31 2017
%\documentclass{article}
%\usepackage{lscape,longtable}
%\begin{document}
\begin{center}
\begin{longtable}{|p{0.9in}|p{0.7in}|p{0.7in}|p{0.7in}|p{0.7in}|}
\caption{Phenotypic variance and covariance estimates with standard errors and 95 percent confidence limits for 13 skin and wool traits}  \\
\hline
\label{tab:sigp}

  Traitpair & Estimate & StdErr & CI95lo & CI95hi \\ 
  \hline
\endfirsthead
\multicolumn{5}{c}%
{\tablename\ \thetable\ -- \textit{Continued from previous page}} \\
\hline
  Traitpair & Estimate & StdErr & CI95lo & CI95hi \\
\hline
\endhead
\hline
\multicolumn{5}{r}{\textit{Continued on next page}} \\
\endfoot
\hline
\endlastfoot

  Cww:Cww & 0.12 & 0.00 & 0.12 & 0.12 \\ 
    Cww:Fnua & -0.13 & 0.07 & -0.27 & 0.01 \\ 
    Cww:Fr & 0.20 & 0.02 & 0.15 & 0.24 \\ 
    Cww:Bctb & 0.03 & 0.01 & 0.01 & 0.05 \\ 
    Cww:NLB & -0.02 & 0.00 & -0.03 & -0.02 \\ 
    Cww:NLW & -0.02 & 0.00 & -0.02 & -0.01 \\ 
    Cww:Dp & 0.10 & 0.04 & 0.03 & 0.17 \\ 
    Cww:Ds & 0.14 & 0.02 & 0.09 & 0.18 \\ 
    Cww:Dps & 0.13 & 0.02 & 0.09 & 0.18 \\ 
    Cww:DpovDs & -0.00 & 0.00 & -0.01 & 0.00 \\ 
    Cww:SDDp & 0.01 & 0.01 & -0.02 & 0.04 \\ 
    Cww:SDDs & 0.01 & 0.01 & -0.00 & 0.02 \\ 
    Cww:SDD & 0.01 & 0.01 & -0.00 & 0.02 \\ 
    Fnua:Cww & -0.13 & 0.07 & -0.27 & 0.01 \\ 
    Fnua:Fnua & 148.64 & 2.63 & 143.50 & 153.79 \\ 
    Fnua:Fr & 24.10 & 0.83 & 22.47 & 25.74 \\ 
    Fnua:Bctb & -0.49 & 0.34 & -1.16 & 0.18 \\ 
    Fnua:NLB & -0.46 & 0.10 & -0.65 & -0.27 \\ 
    Fnua:NLW & -0.27 & 0.09 & -0.45 & -0.10 \\ 
    Fnua:Dp & -11.54 & 1.63 & -14.73 & -8.35 \\ 
    Fnua:Ds & -14.62 & 0.95 & -16.48 & -12.76 \\ 
    Fnua:Dps & -14.71 & 0.94 & -16.55 & -12.86 \\ 
    Fnua:DpovDs & 0.26 & 0.08 & 0.11 & 0.41 \\ 
    Fnua:SDDp & -3.19 & 0.66 & -4.49 & -1.89 \\ 
    Fnua:SDDs & -1.68 & 0.28 & -2.22 & -1.14 \\ 
    Fnua:SDD & -1.86 & 0.28 & -2.41 & -1.32 \\ 
    Fr:Cww & 0.20 & 0.02 & 0.15 & 0.24 \\ 
    Fr:Fnua & 24.10 & 0.83 & 22.47 & 25.74 \\ 
    Fr:Fr & 15.00 & 0.26 & 14.49 & 15.52 \\ 
    Fr:Bctb & -0.16 & 0.11 & -0.38 & 0.05 \\ 
    Fr:NLB & -0.18 & 0.03 & -0.24 & -0.12 \\ 
    Fr:NLW & -0.13 & 0.03 & -0.18 & -0.07 \\ 
    Fr:Dp & -1.70 & 0.52 & -2.71 & -0.69 \\ 
    Fr:Ds & -1.74 & 0.30 & -2.33 & -1.15 \\ 
    Fr:Dps & -1.86 & 0.30 & -2.45 & -1.28 \\ 
    Fr:DpovDs & 0.02 & 0.02 & -0.03 & 0.06 \\ 
    Fr:SDDp & -0.49 & 0.21 & -0.90 & -0.08 \\ 
    Fr:SDDs & -0.20 & 0.09 & -0.37 & -0.03 \\ 
    Fr:SDD & -0.28 & 0.09 & -0.45 & -0.11 \\ 
    Bctb:Cww & 0.03 & 0.01 & 0.01 & 0.05 \\ 
    Bctb:Fnua & -0.49 & 0.34 & -1.16 & 0.18 \\ 
    Bctb:Fr & -0.16 & 0.11 & -0.38 & 0.05 \\ 
    Bctb:Bctb & 2.27 & 0.04 & 2.20 & 2.35 \\ 
    Bctb:NLB & -0.02 & 0.01 & -0.05 & 0.00 \\ 
    Bctb:NLW & -0.02 & 0.01 & -0.04 & -0.00 \\ 
    Bctb:Dp & 2.79 & 0.18 & 2.44 & 3.14 \\ 
    Bctb:Ds & -0.20 & 0.11 & -0.41 & 0.01 \\ 
    Bctb:Dps & -0.07 & 0.11 & -0.28 & 0.13 \\ 
    Bctb:DpovDs & 0.15 & 0.01 & 0.14 & 0.17 \\ 
    Bctb:SDDp & 1.12 & 0.07 & 0.98 & 1.26 \\ 
    Bctb:SDDs & 0.10 & 0.03 & 0.04 & 0.16 \\ 
   Bctb:SDD & 0.17 & 0.03 & 0.11 & 0.23 \\ 
   NLB:Cww & -0.02 & 0.00 & -0.03 & -0.02 \\ 
   NLB:Fnua & -0.46 & 0.10 & -0.65 & -0.27 \\ 
   NLB:Fr & -0.18 & 0.03 & -0.24 & -0.12 \\ 
   NLB:Bctb & -0.02 & 0.01 & -0.05 & 0.00 \\ 
   NLB:NLB & 0.21 & 0.00 & 0.20 & 0.22 \\ 
   NLB:NLW & 0.15 & 0.00 & 0.14 & 0.16 \\ 
   NLB:Dp & 0.30 & 0.06 & 0.18 & 0.41 \\ 
   NLB:Ds & 0.07 & 0.03 & -0.00 & 0.13 \\ 
   NLB:Dps & 0.08 & 0.03 & 0.01 & 0.14 \\ 
   NLB:DpovDs & 0.01 & 0.00 & 0.01 & 0.02 \\ 
   NLB:SDDp & 0.05 & 0.02 & 0.01 & 0.10 \\ 
   NLB:SDDs & 0.02 & 0.01 & 0.00 & 0.04 \\ 
   NLB:SDD & 0.02 & 0.01 & 0.00 & 0.04 \\ 
   NLW:Cww & -0.02 & 0.00 & -0.02 & -0.01 \\ 
   NLW:Fnua & -0.27 & 0.09 & -0.45 & -0.10 \\ 
   NLW:Fr & -0.13 & 0.03 & -0.18 & -0.07 \\ 
   NLW:Bctb & -0.02 & 0.01 & -0.04 & -0.00 \\ 
   NLW:NLB & 0.15 & 0.00 & 0.14 & 0.16 \\ 
   NLW:NLW & 0.17 & 0.00 & 0.17 & 0.18 \\ 
   NLW:Dp & 0.29 & 0.05 & 0.18 & 0.39 \\ 
   NLW:Ds & 0.04 & 0.03 & -0.02 & 0.10 \\ 
   NLW:Dps & 0.05 & 0.03 & -0.01 & 0.12 \\ 
   NLW:DpovDs & 0.01 & 0.00 & 0.01 & 0.02 \\ 
   NLW:SDDp & 0.05 & 0.02 & 0.00 & 0.09 \\ 
   NLW:SDDs & 0.03 & 0.01 & 0.01 & 0.05 \\ 
   NLW:SDD & 0.03 & 0.01 & 0.01 & 0.05 \\ 
   Dp:Cww & 0.10 & 0.04 & 0.03 & 0.17 \\ 
   Dp:Fnua & -11.54 & 1.63 & -14.73 & -8.35 \\ 
   Dp:Fr & -1.70 & 0.52 & -2.71 & -0.69 \\ 
   Dp:Bctb & 2.79 & 0.18 & 2.44 & 3.14 \\ 
   Dp:NLB & 0.30 & 0.06 & 0.18 & 0.41 \\ 
   Dp:NLW & 0.29 & 0.05 & 0.18 & 0.39 \\ 
   Dp:Dp & 11.81 & 0.39 & 11.04 & 12.57 \\ 
   Dp:Ds & 2.60 & 0.23 & 2.15 & 3.05 \\ 
   Dp:Dps & 3.03 & 0.23 & 2.59 & 3.48 \\ 
   Dp:DpovDs & 0.43 & 0.02 & 0.39 & 0.47 \\ 
   Dp:SDDp & 3.16 & 0.16 & 2.85 & 3.47 \\ 
   Dp:SDDs & 0.80 & 0.07 & 0.67 & 0.93 \\ 
   Dp:SDD & 0.97 & 0.07 & 0.84 & 1.10 \\ 
   Ds:Cww & 0.14 & 0.02 & 0.09 & 0.18 \\ 
   Ds:Fnua & -14.62 & 0.95 & -16.48 & -12.76 \\ 
   Ds:Fr & -1.74 & 0.30 & -2.33 & -1.15 \\ 
   Ds:Bctb & -0.20 & 0.11 & -0.41 & 0.01 \\ 
   Ds:NLB & 0.07 & 0.03 & -0.00 & 0.13 \\ 
   Ds:NLW & 0.04 & 0.03 & -0.02 & 0.10 \\ 
   Ds:Dp & 2.60 & 0.23 & 2.15 & 3.05 \\ 
   Ds:Ds & 4.06 & 0.14 & 3.79 & 4.32 \\ 
   Ds:Dps & 4.01 & 0.13 & 3.75 & 4.27 \\ 
   Ds:DpovDs & -0.10 & 0.01 & -0.13 & -0.08 \\ 
   Ds:SDDp & 0.44 & 0.09 & 0.25 & 0.62 \\ 
   Ds:SDDs & 0.43 & 0.04 & 0.36 & 0.51 \\ 
   Ds:SDD & 0.44 & 0.04 & 0.36 & 0.51 \\ 
   Dps:Cww & 0.13 & 0.02 & 0.09 & 0.18 \\ 
   Dps:Fnua & -14.71 & 0.94 & -16.55 & -12.86 \\ 
   Dps:Fr & -1.86 & 0.30 & -2.45 & -1.28 \\ 
   Dps:Bctb & -0.07 & 0.11 & -0.28 & 0.13 \\ 
   Dps:NLB & 0.08 & 0.03 & 0.01 & 0.14 \\ 
   Dps:NLW & 0.05 & 0.03 & -0.01 & 0.12 \\ 
   Dps:Dp & 3.03 & 0.23 & 2.59 & 3.48 \\ 
   Dps:Ds & 4.01 & 0.13 & 3.75 & 4.27 \\ 
   Dps:Dps & 3.99 & 0.13 & 3.73 & 4.25 \\ 
   Dps:DpovDs & -0.08 & 0.01 & -0.10 & -0.06 \\ 
   Dps:SDDp & 0.56 & 0.09 & 0.38 & 0.74 \\ 
   Dps:SDDs & 0.45 & 0.04 & 0.38 & 0.53 \\ 
   Dps:SDD & 0.46 & 0.04 & 0.39 & 0.54 \\ 
   DpovDs:Cww & -0.00 & 0.00 & -0.01 & 0.00 \\ 
   DpovDs:Fnua & 0.26 & 0.08 & 0.11 & 0.41 \\ 
   DpovDs:Fr & 0.02 & 0.02 & -0.03 & 0.06 \\ 
   DpovDs:Bctb & 0.15 & 0.01 & 0.14 & 0.17 \\ 
   DpovDs:NLB & 0.01 & 0.00 & 0.01 & 0.02 \\ 
   DpovDs:NLW & 0.01 & 0.00 & 0.01 & 0.02 \\ 
   DpovDs:Dp & 0.43 & 0.02 & 0.39 & 0.47 \\ 
   DpovDs:Ds & -0.10 & 0.01 & -0.13 & -0.08 \\ 
   DpovDs:Dps & -0.08 & 0.01 & -0.10 & -0.06 \\ 
   DpovDs:DpovDs & 0.03 & 0.00 & 0.03 & 0.03 \\ 
   DpovDs:SDDp & 0.13 & 0.01 & 0.12 & 0.15 \\ 
   DpovDs:SDDs & 0.01 & 0.00 & 0.01 & 0.02 \\ 
   DpovDs:SDD & 0.02 & 0.00 & 0.02 & 0.03 \\ 
   SDDp:Cww & 0.01 & 0.01 & -0.02 & 0.04 \\ 
   SDDp:Fnua & -3.19 & 0.66 & -4.49 & -1.89 \\ 
   SDDp:Fr & -0.49 & 0.21 & -0.90 & -0.08 \\ 
   SDDp:Bctb & 1.12 & 0.07 & 0.98 & 1.26 \\ 
   SDDp:NLB & 0.05 & 0.02 & 0.01 & 0.10 \\ 
   SDDp:NLW & 0.05 & 0.02 & 0.00 & 0.09 \\ 
   SDDp:Dp & 3.16 & 0.16 & 2.85 & 3.47 \\ 
   SDDp:Ds & 0.44 & 0.09 & 0.25 & 0.62 \\ 
   SDDp:Dps & 0.56 & 0.09 & 0.38 & 0.74 \\ 
   SDDp:DpovDs & 0.13 & 0.01 & 0.12 & 0.15 \\ 
   SDDp:SDDp & 1.95 & 0.06 & 1.83 & 2.08 \\ 
   SDDp:SDDs & 0.27 & 0.03 & 0.22 & 0.33 \\ 
   SDDp:SDD & 0.39 & 0.03 & 0.34 & 0.44 \\ 
   SDDs:Cww & 0.01 & 0.01 & -0.00 & 0.02 \\ 
   SDDs:Fnua & -1.68 & 0.28 & -2.22 & -1.14 \\ 
   SDDs:Fr & -0.20 & 0.09 & -0.37 & -0.03 \\ 
   SDDs:Bctb & 0.10 & 0.03 & 0.04 & 0.16 \\ 
   SDDs:NLB & 0.02 & 0.01 & 0.00 & 0.04 \\ 
   SDDs:NLW & 0.03 & 0.01 & 0.01 & 0.05 \\ 
   SDDs:Dp & 0.80 & 0.07 & 0.67 & 0.93 \\ 
   SDDs:Ds & 0.43 & 0.04 & 0.36 & 0.51 \\ 
   SDDs:Dps & 0.45 & 0.04 & 0.38 & 0.53 \\ 
   SDDs:DpovDs & 0.01 & 0.00 & 0.01 & 0.02 \\ 
   SDDs:SDDp & 0.27 & 0.03 & 0.22 & 0.33 \\ 
   SDDs:SDDs & 0.34 & 0.01 & 0.31 & 0.36 \\ 
   SDDs:SDD & 0.33 & 0.01 & 0.31 & 0.36 \\ 
   SDD:Cww & 0.01 & 0.01 & -0.00 & 0.02 \\ 
   SDD:Fnua & -1.86 & 0.28 & -2.41 & -1.32 \\ 
   SDD:Fr & -0.28 & 0.09 & -0.45 & -0.11 \\ 
   SDD:Bctb & 0.17 & 0.03 & 0.11 & 0.23 \\ 
   SDD:NLB & 0.02 & 0.01 & 0.00 & 0.04 \\ 
   SDD:NLW & 0.03 & 0.01 & 0.01 & 0.05 \\ 
   SDD:Dp & 0.97 & 0.07 & 0.84 & 1.10 \\ 
   SDD:Ds & 0.44 & 0.04 & 0.36 & 0.51 \\ 
   SDD:Dps & 0.46 & 0.04 & 0.39 & 0.54 \\ 
   SDD:DpovDs & 0.02 & 0.00 & 0.02 & 0.03 \\ 
   SDD:SDDp & 0.39 & 0.03 & 0.34 & 0.44 \\ 
   SDD:SDDs & 0.33 & 0.01 & 0.31 & 0.36 \\ 
   SDD:SDD & 0.34 & 0.01 & 0.32 & 0.36 \\ 
   \hline
\end{longtable}
\end{center}
%\end{document}


These are only of interest to someone wanting to calculate predicted responses.

\subsection{Phenotypic correlations}
These are the correlation one observes in a flock. They are made up of a genetic and environmental component, so they may differ from a genetic correlation if the environmental correlation differsa. They are presented in Table~\ref{tab:rphen}
% latex table generated in R 3.2.4 by xtable 1.8-2 package
% Thu Sep 21 20:23:02 2017
%\documentclass{article}
%\usepackage{lscape,longtable}
%\begin{document}
\begin{center}
\begin{longtable}{|p{0.9in}|p{0.7in}|p{0.7in}|p{0.7in}|p{0.7in}|}
\caption{Phenotypic correlation estimates with standard errors and 95 percent confidence limits for 13 skin and wool traits}  \\
\hline
\label{tab:rphen}

  Traits & Estimate & StdErr & CI95lo & CI95hi \\ 
  \hline
\endfirsthead
\multicolumn{5}{c}%
{\tablename\ \thetable\ -- \textit{Continued from previous page}} \\
\hline
  Traits & Estimate & StdErr & CI95lo & CI95hi \\
\hline
\endhead
\hline
\multicolumn{5}{r}{\textit{Continued on next page}} \\
\endfoot
\hline
\endlastfoot

Cww:Cww & 1.00 & 0.00 & 1.00 & 1.00 \\ 
  Cww:Fnua & -0.03 & 0.02 & -0.07 & 0.00 \\ 
  Cww:Fr & 0.15 & 0.02 & 0.12 & 0.18 \\ 
  Cww:Bctb & 0.06 & 0.02 & 0.02 & 0.09 \\ 
  Cww:NLB & -0.15 & 0.02 & -0.18 & -0.12 \\ 
  Cww:NLW & -0.13 & 0.02 & -0.16 & -0.10 \\ 
  Cww:Dp & 0.09 & 0.03 & 0.03 & 0.15 \\ 
  Cww:Ds & 0.20 & 0.03 & 0.14 & 0.27 \\ 
  Cww:Dps & 0.20 & 0.03 & 0.14 & 0.26 \\ 
  Cww:DpovDs & -0.06 & 0.03 & -0.12 & 0.00 \\ 
  Cww:SDDp & 0.01 & 0.03 & -0.05 & 0.08 \\ 
  Cww:SDDs & 0.06 & 0.03 & -0.00 & 0.12 \\ 
  Cww:SDD & 0.06 & 0.03 & -0.01 & 0.12 \\ 
  Fnua:Cww & -0.03 & 0.02 & -0.07 & 0.00 \\ 
  Fnua:Fnua & 1.00 & 0.00 & 1.00 & 1.00 \\ 
  Fnua:Fr & 0.51 & 0.01 & 0.48 & 0.54 \\ 
  Fnua:Bctb & -0.03 & 0.02 & -0.06 & 0.01 \\ 
  Fnua:NLB & -0.08 & 0.02 & -0.12 & -0.05 \\ 
  Fnua:NLW & -0.06 & 0.02 & -0.09 & -0.02 \\ 
  Fnua:Dp & -0.24 & 0.03 & -0.30 & -0.17 \\ 
  Fnua:Ds & -0.52 & 0.03 & -0.57 & -0.46 \\ 
  Fnua:Dps & -0.52 & 0.03 & -0.58 & -0.47 \\ 
  Fnua:DpovDs & 0.11 & 0.03 & 0.05 & 0.18 \\ 
  Fnua:SDDp & -0.16 & 0.03 & -0.23 & -0.10 \\ 
  Fnua:SDDs & -0.20 & 0.03 & -0.27 & -0.14 \\ 
  Fnua:SDD & -0.23 & 0.03 & -0.29 & -0.16 \\ 
  Fr:Cww & 0.15 & 0.02 & 0.12 & 0.18 \\ 
  Fr:Fnua & 0.51 & 0.01 & 0.48 & 0.54 \\ 
  Fr:Fr & 1.00 & 0.00 & 1.00 & 1.00 \\ 
  Fr:Bctb & -0.03 & 0.02 & -0.06 & 0.01 \\ 
  Fr:NLB & -0.10 & 0.02 & -0.14 & -0.07 \\ 
  Fr:NLW & -0.08 & 0.02 & -0.12 & -0.05 \\ 
  Fr:Dp & -0.11 & 0.03 & -0.18 & -0.05 \\ 
  Fr:Ds & -0.19 & 0.03 & -0.26 & -0.13 \\ 
  Fr:Dps & -0.21 & 0.03 & -0.27 & -0.15 \\ 
  Fr:DpovDs & 0.02 & 0.03 & -0.04 & 0.09 \\ 
  Fr:SDDp & -0.08 & 0.03 & -0.14 & -0.01 \\ 
  Fr:SDDs & -0.08 & 0.03 & -0.14 & -0.01 \\ 
  Fr:SDD & -0.11 & 0.03 & -0.17 & -0.04 \\ 
  Bctb:Cww & 0.06 & 0.02 & 0.02 & 0.09 \\ 
  Bctb:Fnua & -0.03 & 0.02 & -0.06 & 0.01 \\ 
  Bctb:Fr & -0.03 & 0.02 & -0.06 & 0.01 \\ 
  Bctb:Bctb & 1.00 & 0.00 & 1.00 & 1.00 \\ 
  Bctb:NLB & -0.03 & 0.02 & -0.07 & 0.00 \\ 
  Bctb:NLW & -0.04 & 0.02 & -0.07 & -0.00 \\ 
  Bctb:Dp & 0.42 & 0.02 & 0.37 & 0.47 \\ 
  Bctb:Ds & -0.05 & 0.03 & -0.10 & 0.00 \\ 
  Bctb:Dps & -0.02 & 0.03 & -0.07 & 0.03 \\ 
  Bctb:DpovDs & 0.48 & 0.02 & 0.43 & 0.52 \\ 
  Bctb:SDDp & 0.41 & 0.02 & 0.37 & 0.46 \\ 
  Bctb:SDDs & 0.09 & 0.03 & 0.04 & 0.14 \\ 
  Bctb:SDD & 0.15 & 0.03 & 0.10 & 0.20 \\ 
  NLB:Cww & -0.15 & 0.02 & -0.18 & -0.12 \\ 
  NLB:Fnua & -0.08 & 0.02 & -0.12 & -0.05 \\ 
  NLB:Fr & -0.10 & 0.02 & -0.14 & -0.07 \\ 
  NLB:Bctb & -0.03 & 0.02 & -0.07 & 0.00 \\ 
  NLB:NLB & 1.00 & 0.00 & 1.00 & 1.00 \\ 
  NLB:NLW & 0.79 & 0.01 & 0.77 & 0.81 \\ 
  NLB:Dp & 0.17 & 0.03 & 0.11 & 0.24 \\ 
  NLB:Ds & 0.07 & 0.03 & 0.00 & 0.13 \\ 
  NLB:Dps & 0.08 & 0.03 & 0.01 & 0.14 \\ 
  NLB:DpovDs & 0.14 & 0.03 & 0.07 & 0.20 \\ 
  NLB:SDDp & 0.08 & 0.03 & 0.01 & 0.14 \\ 
  NLB:SDDs & 0.07 & 0.03 & 0.00 & 0.13 \\ 
  NLB:SDD & 0.08 & 0.03 & 0.02 & 0.15 \\ 
  NLW:Cww & -0.13 & 0.02 & -0.16 & -0.10 \\ 
  NLW:Fnua & -0.06 & 0.02 & -0.09 & -0.02 \\ 
  NLW:Fr & -0.08 & 0.02 & -0.12 & -0.05 \\ 
  NLW:Bctb & -0.04 & 0.02 & -0.07 & -0.00 \\ 
  NLW:NLB & 0.79 & 0.01 & 0.77 & 0.81 \\ 
  NLW:NLW & 1.00 & 0.00 & 1.00 & 1.00 \\ 
  NLW:Dp & 0.17 & 0.03 & 0.11 & 0.24 \\ 
  NLW:Ds & 0.04 & 0.03 & -0.02 & 0.11 \\ 
  NLW:Dps & 0.06 & 0.03 & -0.01 & 0.12 \\ 
  NLW:DpovDs & 0.15 & 0.03 & 0.09 & 0.21 \\ 
  NLW:SDDp & 0.07 & 0.03 & 0.00 & 0.13 \\ 
  NLW:SDDs & 0.10 & 0.03 & 0.03 & 0.16 \\ 
  NLW:SDD & 0.11 & 0.03 & 0.04 & 0.17 \\ 
  Dp:Cww & 0.09 & 0.03 & 0.03 & 0.15 \\ 
  Dp:Fnua & -0.24 & 0.03 & -0.30 & -0.17 \\ 
  Dp:Fr & -0.11 & 0.03 & -0.18 & -0.05 \\ 
  Dp:Bctb & 0.42 & 0.02 & 0.37 & 0.47 \\ 
  Dp:NLB & 0.17 & 0.03 & 0.11 & 0.24 \\ 
  Dp:NLW & 0.17 & 0.03 & 0.11 & 0.24 \\ 
  Dp:Dp & 1.00 & 0.00 & 1.00 & 1.00 \\ 
  Dp:Ds & 0.38 & 0.03 & 0.32 & 0.43 \\ 
  Dp:Dps & 0.44 & 0.03 & 0.39 & 0.50 \\ 
  Dp:DpovDs & 0.76 & 0.02 & 0.72 & 0.79 \\ 
  Dp:SDDp & 0.66 & 0.02 & 0.61 & 0.70 \\ 
  Dp:SDDs & 0.40 & 0.03 & 0.34 & 0.46 \\ 
  Dp:SDD & 0.49 & 0.03 & 0.43 & 0.54 \\ 
  Ds:Cww & 0.20 & 0.03 & 0.14 & 0.27 \\ 
  Ds:Fnua & -0.52 & 0.03 & -0.57 & -0.46 \\ 
  Ds:Fr & -0.19 & 0.03 & -0.26 & -0.13 \\ 
  Ds:Bctb & -0.05 & 0.03 & -0.10 & 0.00 \\ 
  Ds:NLB & 0.07 & 0.03 & 0.00 & 0.13 \\ 
  Ds:NLW & 0.04 & 0.03 & -0.02 & 0.11 \\ 
  Ds:Dp & 0.38 & 0.03 & 0.32 & 0.43 \\ 
  Ds:Ds & 1.00 & 0.00 & 1.00 & 1.00 \\ 
  Ds:Dps & 1.00 & 0.00 & 0.99 & 1.00 \\ 
  Ds:DpovDs & -0.31 & 0.03 & -0.37 & -0.25 \\ 
  Ds:SDDp & 0.15 & 0.03 & 0.09 & 0.22 \\ 
  Ds:SDDs & 0.37 & 0.03 & 0.31 & 0.43 \\ 
  Ds:SDD & 0.37 & 0.03 & 0.31 & 0.43 \\ 
  Dps:Cww & 0.20 & 0.03 & 0.14 & 0.26 \\ 
  Dps:Fnua & -0.52 & 0.03 & -0.58 & -0.47 \\ 
  Dps:Fr & -0.21 & 0.03 & -0.27 & -0.15 \\ 
  Dps:Bctb & -0.02 & 0.03 & -0.07 & 0.03 \\ 
  Dps:NLB & 0.08 & 0.03 & 0.01 & 0.14 \\ 
  Dps:NLW & 0.06 & 0.03 & -0.01 & 0.12 \\ 
  Dps:Dp & 0.44 & 0.03 & 0.39 & 0.50 \\ 
  Dps:Ds & 1.00 & 0.00 & 0.99 & 1.00 \\ 
  Dps:Dps & 1.00 & 0.00 & 1.00 & 1.00 \\ 
  Dps:DpovDs & -0.24 & 0.03 & -0.31 & -0.18 \\ 
  Dps:SDDp & 0.20 & 0.03 & 0.14 & 0.26 \\ 
  Dps:SDDs & 0.39 & 0.03 & 0.33 & 0.45 \\ 
  Dps:SDD & 0.40 & 0.03 & 0.34 & 0.46 \\ 
  DpovDs:Cww & -0.06 & 0.03 & -0.12 & 0.00 \\ 
  DpovDs:Fnua & 0.11 & 0.03 & 0.05 & 0.18 \\ 
  DpovDs:Fr & 0.02 & 0.03 & -0.04 & 0.09 \\ 
  DpovDs:Bctb & 0.48 & 0.02 & 0.43 & 0.52 \\ 
  DpovDs:NLB & 0.14 & 0.03 & 0.07 & 0.20 \\ 
  DpovDs:NLW & 0.15 & 0.03 & 0.09 & 0.21 \\ 
  DpovDs:Dp & 0.76 & 0.02 & 0.72 & 0.79 \\ 
  DpovDs:Ds & -0.31 & 0.03 & -0.37 & -0.25 \\ 
  DpovDs:Dps & -0.24 & 0.03 & -0.31 & -0.18 \\ 
  DpovDs:DpovDs & 1.00 & 0.00 & 1.00 & 1.00 \\ 
  DpovDs:SDDp & 0.56 & 0.02 & 0.51 & 0.61 \\ 
  DpovDs:SDDs & 0.15 & 0.03 & 0.09 & 0.21 \\ 
  DpovDs:SDD & 0.24 & 0.03 & 0.17 & 0.30 \\ 
  SDDp:Cww & 0.01 & 0.03 & -0.05 & 0.08 \\ 
  SDDp:Fnua & -0.16 & 0.03 & -0.23 & -0.10 \\ 
  SDDp:Fr & -0.08 & 0.03 & -0.14 & -0.01 \\ 
  SDDp:Bctb & 0.41 & 0.02 & 0.37 & 0.46 \\ 
  SDDp:NLB & 0.08 & 0.03 & 0.01 & 0.14 \\ 
  SDDp:NLW & 0.07 & 0.03 & 0.00 & 0.13 \\ 
  SDDp:Dp & 0.66 & 0.02 & 0.61 & 0.70 \\ 
  SDDp:Ds & 0.15 & 0.03 & 0.09 & 0.22 \\ 
  SDDp:Dps & 0.20 & 0.03 & 0.14 & 0.26 \\ 
  SDDp:DpovDs & 0.56 & 0.02 & 0.51 & 0.61 \\ 
  SDDp:SDDp & 1.00 & 0.00 & 1.00 & 1.00 \\ 
  SDDp:SDDs & 0.34 & 0.03 & 0.28 & 0.40 \\ 
  SDDp:SDD & 0.48 & 0.03 & 0.42 & 0.53 \\ 
  SDDs:Cww & 0.06 & 0.03 & -0.00 & 0.12 \\ 
  SDDs:Fnua & -0.20 & 0.03 & -0.27 & -0.14 \\ 
  SDDs:Fr & -0.08 & 0.03 & -0.14 & -0.01 \\ 
  SDDs:Bctb & 0.09 & 0.03 & 0.04 & 0.14 \\ 
  SDDs:NLB & 0.07 & 0.03 & 0.00 & 0.13 \\ 
  SDDs:NLW & 0.10 & 0.03 & 0.03 & 0.16 \\ 
  SDDs:Dp & 0.40 & 0.03 & 0.34 & 0.46 \\ 
  SDDs:Ds & 0.37 & 0.03 & 0.31 & 0.43 \\ 
  SDDs:Dps & 0.39 & 0.03 & 0.33 & 0.45 \\ 
  SDDs:DpovDs & 0.15 & 0.03 & 0.09 & 0.21 \\ 
  SDDs:SDDp & 0.34 & 0.03 & 0.28 & 0.40 \\ 
  SDDs:SDDs & 1.00 & 0.00 & 1.00 & 1.00 \\ 
  SDDs:SDD & 0.99 & 0.00 & 0.98 & 0.99 \\ 
  SDD:Cww & 0.06 & 0.03 & -0.01 & 0.12 \\ 
  SDD:Fnua & -0.23 & 0.03 & -0.29 & -0.16 \\ 
  SDD:Fr & -0.11 & 0.03 & -0.17 & -0.04 \\ 
  SDD:Bctb & 0.15 & 0.03 & 0.10 & 0.20 \\ 
  SDD:NLB & 0.08 & 0.03 & 0.02 & 0.15 \\ 
  SDD:NLW & 0.11 & 0.03 & 0.04 & 0.17 \\ 
  SDD:Dp & 0.49 & 0.03 & 0.43 & 0.54 \\ 
  SDD:Ds & 0.37 & 0.03 & 0.31 & 0.43 \\ 
  SDD:Dps & 0.40 & 0.03 & 0.34 & 0.46 \\ 
  SDD:DpovDs & 0.24 & 0.03 & 0.17 & 0.30 \\ 
  SDD:SDDp & 0.48 & 0.03 & 0.42 & 0.53 \\ 
  SDD:SDDs & 0.99 & 0.00 & 0.98 & 0.99 \\ 
  SDD:SDD & 1.00 & 0.00 & 1.00 & 1.00 \\ 
   \hline
\end{longtable}
\end{center}
%\end{document}


The phenotypic correlations are mostly of the same sign as the genetic correlations, but of smaller magnitude. Cww is not correlated with SDDp, SDDs, SDD, and the  correlation of Cww with Dp is now slightly positive. Dp and Ds have a phenotypic correlation, but no genetic correlation, so they must therefore have an environmental correlation. We shall see below.
 
\subsection{Environmental correlations}
The environmental correlations are correlation caused by any non-genetic factors which traits have in common. They are presented in Table~\ref{tab:renv}
% latex table generated in R 3.2.4 by xtable 1.8-2 package
% Thu Sep 21 20:46:02 2017
%\documentclass{article}
%\usepackage{lscape,longtable}
%\begin{document}
\begin{center}
\begin{longtable}{|p{0.9in}|p{0.7in}|p{0.7in}|p{0.7in}|p{0.7in}|}
\caption{Environmental correlation estimates with standard errors and 95 percent confidence limits for 13 skin and wool traits}  \\
\hline
\label{tab:renv}

  Traitpair & Estimate & StdErr & CI95lo & CI95hi \\ 
  \hline
\endfirsthead
\multicolumn{5}{c}%
{\tablename\ \thetable\ -- \textit{Continued from previous page}} \\
\hline
  Traitpair & Estimate & StdErr & CI95lo & CI95hi \\
\hline
\endhead
\hline
\multicolumn{5}{r}{\textit{Continued on next page}} \\
\endfoot
\hline
\endlastfoot

Cww:Cww & 1.00 & 0.00 & 1.00 & 1.00 \\ 
  Cww:Fnua & -0.05 & 0.03 & -0.11 & 0.01 \\ 
  Cww:Fr & 0.19 & 0.03 & 0.13 & 0.25 \\ 
  Cww:Bctb & 0.28 & 0.05 & 0.18 & 0.38 \\ 
  Cww:NLB & -0.26 & 0.03 & -0.31 & -0.21 \\ 
  Cww:NLW & -0.22 & 0.03 & -0.27 & -0.16 \\ 
  Cww:Dp & 0.18 & 0.08 & 0.03 & 0.34 \\ 
  Cww:Ds & -0.11 & 0.07 & -0.25 & 0.04 \\ 
  Cww:Dps & -0.09 & 0.07 & -0.23 & 0.05 \\ 
  Cww:DpovDs & 0.40 & 0.13 & 0.14 & 0.66 \\ 
  Cww:SDDp & 0.09 & 0.08 & -0.06 & 0.24 \\ 
  Cww:SDDs & -0.12 & 0.07 & -0.26 & 0.02 \\ 
  Cww:SDD & -0.11 & 0.07 & -0.25 & 0.04 \\ 
  Fnua:Cww & -0.05 & 0.03 & -0.11 & 0.01 \\ 
  Fnua:Fnua & 1.00 & 0.00 & 1.00 & 1.00 \\ 
  Fnua:Fr & 0.52 & 0.02 & 0.47 & 0.56 \\ 
  Fnua:Bctb & -0.14 & 0.05 & -0.25 & -0.04 \\ 
  Fnua:NLB & -0.16 & 0.03 & -0.22 & -0.11 \\ 
  Fnua:NLW & -0.14 & 0.03 & -0.20 & -0.09 \\ 
  Fnua:Dp & -0.42 & 0.07 & -0.56 & -0.28 \\ 
  Fnua:Ds & -0.40 & 0.06 & -0.51 & -0.29 \\ 
  Fnua:Dps & -0.41 & 0.05 & -0.52 & -0.31 \\ 
  Fnua:DpovDs & -0.20 & 0.12 & -0.45 & 0.04 \\ 
  Fnua:SDDp & -0.24 & 0.07 & -0.37 & -0.10 \\ 
  Fnua:SDDs & -0.11 & 0.06 & -0.24 & 0.02 \\ 
  Fnua:SDD & -0.15 & 0.07 & -0.28 & -0.02 \\ 
  Fr:Cww & 0.19 & 0.03 & 0.13 & 0.25 \\ 
  Fr:Fnua & 0.52 & 0.02 & 0.47 & 0.56 \\ 
  Fr:Fr & 1.00 & 0.00 & 1.00 & 1.00 \\ 
  Fr:Bctb & -0.08 & 0.05 & -0.19 & 0.02 \\ 
  Fr:NLB & -0.17 & 0.03 & -0.22 & -0.11 \\ 
  Fr:NLW & -0.15 & 0.03 & -0.21 & -0.10 \\ 
  Fr:Dp & -0.11 & 0.07 & -0.25 & 0.04 \\ 
  Fr:Ds & -0.23 & 0.06 & -0.36 & -0.11 \\ 
  Fr:Dps & -0.24 & 0.06 & -0.36 & -0.12 \\ 
  Fr:DpovDs & 0.11 & 0.12 & -0.12 & 0.34 \\ 
  Fr:SDDp & -0.05 & 0.07 & -0.19 & 0.08 \\ 
  Fr:SDDs & 0.02 & 0.07 & -0.11 & 0.15 \\ 
  Fr:SDD & -0.01 & 0.06 & -0.14 & 0.11 \\ 
  Bctb:Cww & 0.28 & 0.05 & 0.18 & 0.38 \\ 
  Bctb:Fnua & -0.14 & 0.05 & -0.25 & -0.04 \\ 
  Bctb:Fr & -0.08 & 0.05 & -0.19 & 0.02 \\ 
  Bctb:Bctb & 1.00 & 0.00 & 1.00 & 1.00 \\ 
  Bctb:NLB & -0.14 & 0.05 & -0.23 & -0.04 \\ 
  Bctb:NLW & -0.27 & 0.05 & -0.37 & -0.17 \\ 
  Bctb:Dp & -0.18 & 0.15 & -0.48 & 0.11 \\ 
  Bctb:Ds & 0.12 & 0.12 & -0.12 & 0.35 \\ 
  Bctb:Dps & 0.09 & 0.12 & -0.13 & 0.32 \\ 
  Bctb:DpovDs & -0.38 & 0.25 & -0.86 & 0.11 \\ 
  Bctb:SDDp & -0.52 & 0.17 & -0.85 & -0.19 \\ 
  Bctb:SDDs & -0.32 & 0.13 & -0.58 & -0.06 \\ 
  Bctb:SDD & -0.39 & 0.14 & -0.66 & -0.11 \\ 
  NLB:Cww & -0.26 & 0.03 & -0.31 & -0.21 \\ 
  NLB:Fnua & -0.16 & 0.03 & -0.22 & -0.11 \\ 
  NLB:Fr & -0.17 & 0.03 & -0.22 & -0.11 \\ 
  NLB:Bctb & -0.14 & 0.05 & -0.23 & -0.04 \\ 
  NLB:NLB & 1.00 & 0.00 & 1.00 & 1.00 \\ 
  NLB:NLW & 0.74 & 0.01 & 0.72 & 0.77 \\ 
  NLB:Dp & 0.04 & 0.07 & -0.09 & 0.18 \\ 
  NLB:Ds & 0.17 & 0.06 & 0.05 & 0.30 \\ 
  NLB:Dps & 0.17 & 0.06 & 0.05 & 0.29 \\ 
  NLB:DpovDs & -0.15 & 0.12 & -0.37 & 0.08 \\ 
  NLB:SDDp & -0.09 & 0.07 & -0.22 & 0.04 \\ 
  NLB:SDDs & 0.07 & 0.06 & -0.05 & 0.19 \\ 
  NLB:SDD & 0.06 & 0.06 & -0.07 & 0.18 \\ 
  NLW:Cww & -0.22 & 0.03 & -0.27 & -0.16 \\ 
  NLW:Fnua & -0.14 & 0.03 & -0.20 & -0.09 \\ 
  NLW:Fr & -0.15 & 0.03 & -0.21 & -0.10 \\ 
  NLW:Bctb & -0.27 & 0.05 & -0.37 & -0.17 \\ 
  NLW:NLB & 0.74 & 0.01 & 0.72 & 0.77 \\ 
  NLW:NLW & 1.00 & 0.00 & 1.00 & 1.00 \\ 
  NLW:Dp & 0.02 & 0.07 & -0.11 & 0.16 \\ 
  NLW:Ds & 0.18 & 0.06 & 0.06 & 0.31 \\ 
  NLW:Dps & 0.18 & 0.06 & 0.05 & 0.30 \\ 
  NLW:DpovDs & -0.20 & 0.12 & -0.43 & 0.04 \\ 
  NLW:SDDp & -0.14 & 0.07 & -0.27 & -0.00 \\ 
  NLW:SDDs & 0.04 & 0.06 & -0.08 & 0.16 \\ 
  NLW:SDD & 0.02 & 0.06 & -0.10 & 0.14 \\ 
  Dp:Cww & 0.18 & 0.08 & 0.03 & 0.34 \\ 
  Dp:Fnua & -0.42 & 0.07 & -0.56 & -0.28 \\ 
  Dp:Fr & -0.11 & 0.07 & -0.25 & 0.04 \\ 
  Dp:Bctb & -0.18 & 0.15 & -0.48 & 0.11 \\ 
  Dp:NLB & 0.04 & 0.07 & -0.09 & 0.18 \\ 
  Dp:NLW & 0.02 & 0.07 & -0.11 & 0.16 \\ 
  Dp:Dp & 1.00 & 0.00 & 1.00 & 1.00 \\ 
  Dp:Ds & 0.78 & 0.07 & 0.65 & 0.91 \\ 
  Dp:Dps & 0.81 & 0.06 & 0.69 & 0.93 \\ 
  Dp:DpovDs & 0.65 & 0.08 & 0.49 & 0.80 \\ 
  Dp:SDDp & 0.46 & 0.06 & 0.34 & 0.58 \\ 
  Dp:SDDs & 0.45 & 0.07 & 0.32 & 0.59 \\ 
  Dp:SDD & 0.50 & 0.06 & 0.38 & 0.63 \\ 
  Ds:Cww & -0.11 & 0.07 & -0.25 & 0.04 \\ 
  Ds:Fnua & -0.40 & 0.06 & -0.51 & -0.29 \\ 
  Ds:Fr & -0.23 & 0.06 & -0.36 & -0.11 \\ 
  Ds:Bctb & 0.12 & 0.12 & -0.12 & 0.35 \\ 
  Ds:NLB & 0.17 & 0.06 & 0.05 & 0.30 \\ 
  Ds:NLW & 0.18 & 0.06 & 0.06 & 0.31 \\ 
  Ds:Dp & 0.78 & 0.07 & 0.65 & 0.91 \\ 
  Ds:Ds & 1.00 & 0.00 & 1.00 & 1.00 \\ 
  Ds:Dps & 1.00 & 0.00 & 0.99 & 1.01 \\ 
  Ds:DpovDs & 0.06 & 0.14 & -0.21 & 0.33 \\ 
  Ds:SDDp & 0.33 & 0.07 & 0.18 & 0.47 \\ 
  Ds:SDDs & 0.22 & 0.07 & 0.09 & 0.34 \\ 
  Ds:SDD & 0.25 & 0.07 & 0.12 & 0.38 \\ 
  Dps:Cww & -0.09 & 0.07 & -0.23 & 0.05 \\ 
  Dps:Fnua & -0.41 & 0.05 & -0.52 & -0.31 \\ 
  Dps:Fr & -0.24 & 0.06 & -0.36 & -0.12 \\ 
  Dps:Bctb & 0.09 & 0.12 & -0.13 & 0.32 \\ 
  Dps:NLB & 0.17 & 0.06 & 0.05 & 0.29 \\ 
  Dps:NLW & 0.18 & 0.06 & 0.05 & 0.30 \\ 
  Dps:Dp & 0.81 & 0.06 & 0.69 & 0.93 \\ 
  Dps:Ds & 1.00 & 0.00 & 0.99 & 1.01 \\ 
  Dps:Dps & 1.00 & 0.00 & 1.00 & 1.00 \\ 
  Dps:DpovDs & 0.11 & 0.13 & -0.15 & 0.36 \\ 
  Dps:SDDp & 0.34 & 0.07 & 0.20 & 0.48 \\ 
  Dps:SDDs & 0.23 & 0.06 & 0.11 & 0.36 \\ 
  Dps:SDD & 0.27 & 0.06 & 0.15 & 0.40 \\ 
  DpovDs:Cww & 0.40 & 0.13 & 0.14 & 0.66 \\ 
  DpovDs:Fnua & -0.20 & 0.12 & -0.45 & 0.04 \\ 
  DpovDs:Fr & 0.11 & 0.12 & -0.12 & 0.34 \\ 
  DpovDs:Bctb & -0.38 & 0.25 & -0.86 & 0.11 \\ 
  DpovDs:NLB & -0.15 & 0.12 & -0.37 & 0.08 \\ 
  DpovDs:NLW & -0.20 & 0.12 & -0.43 & 0.04 \\ 
  DpovDs:Dp & 0.65 & 0.08 & 0.49 & 0.80 \\ 
  DpovDs:Ds & 0.06 & 0.14 & -0.21 & 0.33 \\ 
  DpovDs:Dps & 0.11 & 0.13 & -0.15 & 0.36 \\ 
  DpovDs:DpovDs & 1.00 & 0.00 & 1.00 & 1.00 \\ 
  DpovDs:SDDp & 0.31 & 0.11 & 0.10 & 0.52 \\ 
  DpovDs:SDDs & 0.51 & 0.13 & 0.27 & 0.76 \\ 
  DpovDs:SDD & 0.54 & 0.12 & 0.30 & 0.78 \\ 
  SDDp:Cww & 0.09 & 0.08 & -0.06 & 0.24 \\ 
  SDDp:Fnua & -0.24 & 0.07 & -0.37 & -0.10 \\ 
  SDDp:Fr & -0.05 & 0.07 & -0.19 & 0.08 \\ 
  SDDp:Bctb & -0.52 & 0.17 & -0.85 & -0.19 \\ 
  SDDp:NLB & -0.09 & 0.07 & -0.22 & 0.04 \\ 
  SDDp:NLW & -0.14 & 0.07 & -0.27 & -0.00 \\ 
  SDDp:Dp & 0.46 & 0.06 & 0.34 & 0.58 \\ 
  SDDp:Ds & 0.33 & 0.07 & 0.18 & 0.47 \\ 
  SDDp:Dps & 0.34 & 0.07 & 0.20 & 0.48 \\ 
  SDDp:DpovDs & 0.31 & 0.11 & 0.10 & 0.52 \\ 
  SDDp:SDDp & 1.00 & 0.00 & 1.00 & 1.00 \\ 
  SDDp:SDDs & 0.32 & 0.07 & 0.18 & 0.45 \\ 
  SDDp:SDD & 0.45 & 0.06 & 0.32 & 0.57 \\ 
  SDDs:Cww & -0.12 & 0.07 & -0.26 & 0.02 \\ 
  SDDs:Fnua & -0.11 & 0.06 & -0.24 & 0.02 \\ 
  SDDs:Fr & 0.02 & 0.07 & -0.11 & 0.15 \\ 
  SDDs:Bctb & -0.32 & 0.13 & -0.58 & -0.06 \\ 
  SDDs:NLB & 0.07 & 0.06 & -0.05 & 0.19 \\ 
  SDDs:NLW & 0.04 & 0.06 & -0.08 & 0.16 \\ 
  SDDs:Dp & 0.45 & 0.07 & 0.32 & 0.59 \\ 
  SDDs:Ds & 0.22 & 0.07 & 0.09 & 0.34 \\ 
  SDDs:Dps & 0.23 & 0.06 & 0.11 & 0.36 \\ 
  SDDs:DpovDs & 0.51 & 0.13 & 0.27 & 0.76 \\ 
  SDDs:SDDp & 0.32 & 0.07 & 0.18 & 0.45 \\ 
  SDDs:SDDs & 1.00 & 0.00 & 1.00 & 1.00 \\ 
  SDDs:SDD & 0.99 & 0.01 & 0.97 & 1.00 \\ 
  SDD:Cww & -0.11 & 0.07 & -0.25 & 0.04 \\ 
  SDD:Fnua & -0.15 & 0.07 & -0.28 & -0.02 \\ 
  SDD:Fr & -0.01 & 0.06 & -0.14 & 0.11 \\ 
  SDD:Bctb & -0.39 & 0.14 & -0.66 & -0.11 \\ 
  SDD:NLB & 0.06 & 0.06 & -0.07 & 0.18 \\ 
  SDD:NLW & 0.02 & 0.06 & -0.10 & 0.14 \\ 
  SDD:Dp & 0.50 & 0.06 & 0.38 & 0.63 \\ 
  SDD:Ds & 0.25 & 0.07 & 0.12 & 0.38 \\ 
  SDD:Dps & 0.27 & 0.06 & 0.15 & 0.40 \\ 
  SDD:DpovDs & 0.54 & 0.12 & 0.30 & 0.78 \\ 
  SDD:SDDp & 0.45 & 0.06 & 0.32 & 0.57 \\ 
  SDD:SDDs & 0.99 & 0.01 & 0.97 & 1.00 \\ 
  SDD:SDD & 1.00 & 0.00 & 1.00 & 1.00 \\ 
   \hline
\end{longtable}
\end{center}
%\end{document}


We see that Cww has a small positive environmental correlation with Dp, and a small negative environmental correlation with Ds.  We also see that Dp and Ds have a stron positive environmental correlation, as predicted above from noting a phenotypic correlation but no genetic correlation.  One would expect something like good nutrition to affect all follicles positively. 

The other relationship that shows an inversion of sign is Cww with fertility (NLB and NLW) and birthcoat. There is a positive genetic correlation and a negative environmental correlation with fertility, and the other way around with birthcoat. It would seem that there are fitness issues with Cww, but they are complicated and the net effect may be a balance of competing forces.

\section{Discussion}
Genetic parameters are not the final word on what happens in selected flocks. One can use parameters to predict genetic change under selection. For single trait selection and predicting response in the trait selected, they work quite well. You only need a heritability and a phenotypic variance for that. 

However for multi-trait selection and for responses in traits other than those dselected on, the reliability of predictions is less than one would wish. 

When it comes to the occurrence of rare individuals showing  characteristics resembling primitive or ancestral breeds of sheep, genetic parameters tell us nothing useful. Atavism is about what happens when normal quantitative genetics breaks down . Unusual individuals are just that - outside of the normal distribution for one or more traits. The interesting thing is that they always seem to be on the ancestral side of the distribution. Distributions always skew towards the wild type. There is nothing in quantitative genetics about that. 

\clearpage
\begin{thebibliography}{99}

\bibitem{brow:68}
Brown, G.H., and Turner, Helen Newton. (1968) Response to selection in Australian Merino sheep. II. Estimates of phenotypic and genetic parameters for some production traits in Merino ewes and an analysis of the possible effects of selection on them. Aust. J. Agric. Res. 19:303-22

\bibitem{cart:43}
Carter, H.B. (1943) Studies in the biology of the skin and fleece of sheep. CSIRO (Aust) Bull. No. 164

\bibitem{cart:68}
Carter,H.B. (1968) Comparative Fleece Analysis Data for Domestic Sheep. The Principal Fleece Staple Values of Some Recognised Breeds. Agricultural Research Council, 1968
 
\bibitem{fras:53}
Fraser, A.S. (1953) A note on the growth of Rex and Angora coats. J. Genetics 521:237-42

\bibitem{fras:60}
Fraser A.S and Short B.F. (1960) The Biology of the Fleece. Animal Research Laboratories Technical Paper No 3. CSIRO Melbourne 1960.

\bibitem{jack:75}
Jackson, N., Nay, T, and Turner, Helen Newton (1975) Response to selection in Australian Merino sheep. VII Phenotypic and genetic parameters for some wool follicle characteristics and their correlation with wool and body traits. Aust. J. Agric. Res. 26:937-57

\bibitem{jack:15}
Jackson, N. (2015) Genetic relationship betweeen skin and wool traits in Merino sheep. Incomplete manuscript. 27 Oct 2015

\bibitem{jack:15b}
Jackson, N. (2015b) An Overview of the R Package {\em dmm}. From URL http://cran.r-project.org/package=dmm  Or URL https://github.com/cran/dmm

\bibitem{jack:16}
Jackson, N. and Watts, J.E. (2016) Staple crimp formation in the fleece of Merino sheep. Unpublished manuscript, 18 May 2016.


\bibitem{jack:17}
Jackson, N. (2017) Genetics of primary and secondary fibre diameters and densities in Merino sheep. URL https://github.com/nevillejackson/atavistic-sheep/mev-rewrite/supplementary/genetic-parameters/psparam.pdf

\bibitem{jack:17a}
Jackson, N. (2017) Components of clean wool weight, a restatement incorporating variance of fibre diameter. URL https://github.com/nevillejackson/atavistic-sheep/mev-rewrite/supplementary/cwwcomponents/components.pdf

\bibitem{jack:90}
Jackson, N., Maddocks, I.G., Lax, J., Moore, G.P.M. and Watts, J.E. (1990) Merino Evolution, Skin Characteristics, and Fleece Quality. URL https://github.com/nevillejackson/atavistic-sheep/mev/evol.pdf 

\bibitem{mass:07}
Massy, C.(2007) The Australian Merino. Random House, Sydney, 2007
\bibitem{moor:89}
Moore G.P.M., Jackson, N., and Lax, J. (1989) Evidence of a unique developmental mechanism specifying bot wool follicle density and fibre size in sheep selected for single skin and fleece characters. Genet. Res. Camb. 53:57-62

\bibitem{moor:98}
Moore, G.P.M., Jackson, N., Isaacs, K., and Brown, G (1998) J. Theoretical Biology 191:87-94


\bibitem{nayj:73}
Nay, T. and Jackson, N. (1973) Effect of changes in nutritional level on the depth and curvature of wool follicles in Australian Merino sheep. Aust. J. Agric. Res. 24:439-447

\bibitem{onio:62}
Onions, W.J. (1962) Wool: an introduction to its properties, varieties, uses
     and production. Ernest Benn limited, London, 1962

\bibitem{rprog:13}
R Core Team (2013). R: A language and environment for statistical
  computing. R Foundation for Statistical Computing, Vienna, Austria.
  ISBN 3-900051-07-0, URL http://www.R-project.org/.

\bibitem{ryde:81}
Ryder, M.L. (1981) A Survey of European Primitive Breeds of Sheep. Ann. Genet. Sel. anim. 13(4):381-418

\bibitem{ryde:92}
Ryder, M.L. (1992) The interaction between biological and technological change during the development of different fleece types in sheep. Anthropozoologica 16:131-140

\bibitem{turn:56} 
Turner, Helen Newton (1956) Anim. Breed. Abstr. 24:87-118

\bibitem{turn:58}
Turner, Helen Newton(1958) Aust. J. Agric. Res. 9:521-52

\bibitem{turn:53}
Turner, Helen Newton, Hayman, R.H., Riches, J.H., Roberts, N.F., and Wilson, L.T. (1953) Physical definition of sheep and their fleece for breeding and husbandry studies: with particular reference to Merino sheep. CSIRO Div. Anim. Hlth. Prod. Div. Rept. No. 4 (Ser SW-2 mimeo)

\bibitem{turn:69}
Turner, Helen Newton (1969) Quantitative Genetics and Sheep Breeding. Macmillan, Melbourne, 1969

\bibitem{turn:70}
Turner, Helen Newton, Brooker M.G. and Dolling, C.H.S (1970) Response to selection in Australian Merino sheep. III Single character selection for high and low values of wool weight and its components. Aust.J.Agric.Res. 21:955-84

\bibitem{vonb:48}
Von Bergen, W. and Mauersberger, H.R.(1948) American Wool Handbook. 2nd ed. Barnes, New York.

\bibitem{watt:17}
Watts, J.E. (2017) Personal communication.
\end{thebibliography}
\end{document}
