%
% Draft  document psparam.tex
% Summarizes best available estimates of heritabilities and genetic correlations
% for Dp, Ds, Np, Ns, Ns/Np, and some other wool traits.
%
 
\documentclass[titlepage]{article}  % Latex2e
\usepackage{graphicx,lscape,subfigure}
\usepackage{bm,longtable}
\usepackage{textcomp}
 

\title{Genetics of primary and secondary fibre diameters and densities in Merino sheep}
\author{Neville Jackson}
\date{18 Sep 2017} 

 
\begin{document} 


 
\maketitle      
\tableofcontents

$\newcommand{\E}{\mathrm{E}}$
$\newcommand{\Var}{\mathrm{Var}}$
$\newcommand{\Cov}{\mathrm{Cov}}$ 
$\newcommand{\SD}{\mathrm{SD}}$ 

\clearpage
\section{Introduction} 
In the document Jackson etal(1990)~\cite{jack:90} some genetic parameter estimates were presented ( see Table 8 of Jackson etal(1990)~\cite{jack:90}) for diameters of primary and secondary fibres and a number of  other important wool characteristics. These were preliminary estimates obtained by the now outdated analysis of variance method for partitioning variances between and within half-sib families. 

A modern method of obtaining these parameters would be to estimate additive genetic variance components using maximum likelihood methods and incorating the full additive genetic relationship matrix  among all individuals, computed from a pedigree going back to the base generation of the flock. 

This analysis has been done on the exact same dataset for a comprehensive set of 48 traits (Jackson (2015)~\cite{jack:15}). What we present here is parameter estimates for a small subset of these traits which are relevant to the issues of Merino evolution and atavism.

\section{Materials and methods}
The sheep and measurements were as described in Jackson (2015)~\cite{jack:15}.
\subsection{Traits}
The subset of traits covered here are summarized in Table~\ref{tab:traits}
%\documentclass{article}
%\usepackage{lscape,longtable}
%\begin{document}
\begin{center}
\begin{landscape}
\begin{longtable}{p{1.5in}|p{0.8in}|p{1.5in}|p{1.0in}|p{2.5in}}
\caption{Definition of traits measured}  \\
\hline
\label{tab:traits}
    Trait name & Abbreviation  & Units & Age measured  &  Description \\ 
\hline
\endfirsthead
\multicolumn{5}{c}%
{\tablename\ \thetable\ -- \textit{Continued from previous page}} \\
\hline
    Trait name & Abbreviation  & Units & Age measured  &  Description \\ 
\hline
\endhead
\hline
\multicolumn{5}{r}{\textit{Continued on next page}} \\
\endfoot
\hline
\endlastfoot
%\env{longtable}[p{1.5in}|p{0.8in}|p{1.5in}|p{1.0in}|p{2.5in}]
 Clean wool weight & Cww & Kg & 14 months & Weight of clean fibre at 16\% regain \\
 Follicle number per unit area & Fnua & no per $mm_{2}$ & 14 months & No of primary and secondary follicles per $mm_{2}$ from skin biopsy \\
 Follicle $S/P$ ratio & Fr & no units & 14 months & Ratio of no of primary to no of secondary follicles from skin biopsy \\
  Birthcoat score back & Bctb & score 1-6 (1=no halo hairs on mid backline, 6=dense halo hairs) & day of birth & Score for density of halo hairs on mid backline on day of birth \\
  No of lambs born & NLB & no & day of birth & Number of lambs in litter at birth \\
  No of lambs weaned & NLW & no & approx 4 months & Number of lambs in litter at weaning \\
  Mean diameter of primaries & Dp & microns & 14 months & Mean diameter of primary fibres from biopsy and horizontal section \\
  Mean diameter of secondaries & Ds & microns & 14 months & Mean diameter of secondary fibres from biopsy and horizontal section \\
  Mean diameter of primaries and secondaries & Dps & microns & 14 months & Mean diameter of primary and secondary fibres from biopsy and horizontal section \\
  Primary to secondary diameter ratio & DpovDs & no units & 14 months & Ratio of mean diameter of primary fibres to mean diameter of secondary fibres \\

\end{longtable}
\end{landscape}
\end{center}
%\end{document}


\subsection{Statistical Methods}
The statistical techniques used for estimation of parameters for these data are covered in Jackson (2015)~\cite{jack:15}.  The software used is called {\em dmm}, which runs as a package under the R statistical language ~\cite{}. {\em dmm} is documented in a user's guide (Jackson(2015b)~\cite{jack:15b}).

\section{Results}

\section{Discussion}

\clearpage
\begin{thebibliography}{99}

\bibitem{brow:68}
Brown, G.H., and Turner, Helen Newton. (1968) Response to selection in Australian Merino sheep. II. Estimates of phenotypic and genetic parameters for some production traits in Merino ewes and an analysis of the possible effects of selection on them. Aust. J. Agric. Res. 19:303-22

\bibitem{cart:43}
Carter, H.B. (1943) Studies in the biology of the skin and fleece of sheep. CSIRO (Aust) Bull. No. 164

\bibitem{cart:68}
Carter,H.B. (1968) Comparative Fleece Analysis Data for Domestic Sheep. The Principal Fleece Staple Values of Some Recognised Breeds. Agricultural Research Council, 1968
 
\bibitem{fras:53}
Fraser, A.S. (1953) A note on the growth of Rex and Angora coats. J. Genetics 521:237-42

\bibitem{fras:60}
Fraser A.S and Short B.F. (1960) The Biology of the Fleece. Animal Research Laboratories Technical Paper No 3. CSIRO Melbourne 1960.

\bibitem{jack:75}
Jackson, N., Nay, T, and Turner, Helen Newton (1975) Response to selection in Australian Merino sheep. VII Phenotypic and genetic parameters for some wool follicle characteristics and their correlation with wool and body traits. Aust. J. Agric. Res. 26:937-57

\bibitem{jack:15}
Jackson, N. (2015) Genetic relationship betweeen skin and wool traits in Merino sheep. Incomplete manuscript. 27 Oct 2015

\bibitem{jack:15b}
Jackson, N. (2015b) An Overview of the R Package {\em dmm}. From URL http://cran.r-project.org/package=dmm  Or URL https://github.com/cran/dmm

\bibitem{jack:16}
Jackson, N. and Watts, J.E. (2016) Staple crimp formation in the fleece of Merino sheep. Unpublished manuscript, 18 May 2016.


\bibitem{jack:17}
Jackson, N. (2017) Genetics of primary and secondary fibre diameters and densities in Merino sheep. URL https://github.com/nevillejackson/atavistic-sheep/mev-rewrite/supplementary/genetic-parameters/psparam.pdf

\bibitem{jack:17a}
Jackson, N. (2017) Components of clean wool weight, a restatement incorporating variance of fibre diameter. URL https://github.com/nevillejackson/atavistic-sheep/mev-rewrite/supplementary/cwwcomponents/components.pdf

\bibitem{jack:90}
Jackson, N., Maddocks, I.G., Lax, J., Moore, G.P.M. and Watts, J.E. (1990) Merino Evolution, Skin Characteristics, and Fleece Quality. URL https://github.com/nevillejackson/atavistic-sheep/mev/evol.pdf 

\bibitem{mass:07}
Massy, C.(2007) The Australian Merino. Random House, Sydney, 2007
\bibitem{moor:89}
Moore G.P.M., Jackson, N., and Lax, J. (1989) Evidence of a unique developmental mechanism specifying bot wool follicle density and fibre size in sheep selected for single skin and fleece characters. Genet. Res. Camb. 53:57-62

\bibitem{moor:98}
Moore, G.P.M., Jackson, N., Isaacs, K., and Brown, G (1998) J. Theoretical Biology 191:87-94


\bibitem{nayj:73}
Nay, T. and Jackson, N. (1973) Effect of changes in nutritional level on the depth and curvature of wool follicles in Australian Merino sheep. Aust. J. Agric. Res. 24:439-447

\bibitem{onio:62}
Onions, W.J. (1962) Wool: an introduction to its properties, varieties, uses
     and production. Ernest Benn limited, London, 1962

\bibitem{rprog:13}
R Core Team (2013). R: A language and environment for statistical
  computing. R Foundation for Statistical Computing, Vienna, Austria.
  ISBN 3-900051-07-0, URL http://www.R-project.org/.

\bibitem{ryde:81}
Ryder, M.L. (1981) A Survey of European Primitive Breeds of Sheep. Ann. Genet. Sel. anim. 13(4):381-418

\bibitem{ryde:92}
Ryder, M.L. (1992) The interaction between biological and technological change during the development of different fleece types in sheep. Anthropozoologica 16:131-140

\bibitem{turn:56} 
Turner, Helen Newton (1956) Anim. Breed. Abstr. 24:87-118

\bibitem{turn:58}
Turner, Helen Newton(1958) Aust. J. Agric. Res. 9:521-52

\bibitem{turn:53}
Turner, Helen Newton, Hayman, R.H., Riches, J.H., Roberts, N.F., and Wilson, L.T. (1953) Physical definition of sheep and their fleece for breeding and husbandry studies: with particular reference to Merino sheep. CSIRO Div. Anim. Hlth. Prod. Div. Rept. No. 4 (Ser SW-2 mimeo)


\bibitem{turn:70}
Turner, Helen Newton, Brooker M.G. and Dolling, C.H.S (1970) Response to selection in Australian Merino sheep. III Single character selection for high and low values of wool weight and its components. Aust.J.Agric.Res. 21:955-84

\bibitem{vonb:48}
Von Bergen, W. and Mauersberger, H.R.(1948) American Wool Handbook. 2nd ed. Barnes, New York.

\bibitem{watt:17}
Watts, J.E. (2017) Personal communication.
\end{thebibliography}
\end{document}
